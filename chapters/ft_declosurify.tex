\label{ft_declosurify}

This chapter presents the core of our work. Section~\ref{ft_declosurify_overview} will give us a
thorough overview of \textsc{ft-declosurify} and section~\ref{ft_declosurify_impl} will describe the major
\textsc{ft-declosurify} implementation details.

\section{\textsc{ft-declosurify} Overview}
\label{ft_declosurify_overview}

For understanding at a high-level what \textsc{ft-declosurify} does let's see how an
example expression \sv{List(1, 2, 3).map(\_ + 1)} is translated from both the scalac
and the \textsc{ft-declosurify} points of view. \sv{List(1, 2, 3).map(\_ + 1)} is a shortcut
for \sv{List(1, 2, 3).map(x => x + 1)}, where \sv{x => x + 1} is an anonymous function
-closure- that returns its argument incremented by one.
Applying that function to the \sv{List(1, 2, 3)} will result in a new list \sv{List(2, 3, 4)}. 
Since Scala code is translated into Java bytecode at the end and since
Java  doesn't support any notion of functions inherently, this anonymous
function should be translated somehow in constructs that are supported by the
Java bytecode. This trivial Scala expression \sv{List(1, 2, 3).map(\_ + 1)} is translated internally 
by scalac to Figure~\ref{scalac_map_internal_repr}'s code\footnote{The code listing below as well as most of the following listings are slightly abbreviated for readability reasons}.

\begin{figure}
\begin{scalaCode}
immutable.this.List.apply[Int](Array[Int]{1, 2, 3}).map[Int, List[Int]]({
      final class $anonfun extends scala.runtime.AbstractFunction1[Int,Int] with Serializable {
        def <init>(): anonymous class $anonfun = {
          $anonfun.super.<init>();
          ()
        };
        final def apply(x$1: Int): Int = x$1.+(1)
      };
      (new $anonfun(): Int => Int)
    }, immutable.this.List.canBuildFrom[Int]())
\end{scalaCode}
\caption[scalac's internal representation of \sv{List(1, 2, 3).map(\_ + 1)}]{scalac's internal representation of \sv{List(1, 2, 3).map(\_ + 1)}}
\label{scalac_map_internal_repr}
\end{figure}

We can see that scalac converts the \sv{\_ + 1} function into a block of code
(piece of code between two braces), where a class, called \texttt{\$anonfun} here, is defined.
That class extends the \sv{AbstractFunction1[Int,Int]} class, which is an abstract
class that represents the functions that accept one integer argument and return
another integer. Inside the class, an \texttt{apply} method is defined which is called
whenever we apply a class object to one integer argument. The \texttt{apply} body
returns  its argument incremented by one,  e.g., \sv{val a1 = new \$anonfun();
a1(5);} returns $6$. Just after the class definition, scalac creates a new object
of this class and this is what is actually returned from that block of code.
Eventually, the \sv{\_ + 1} is substituted by an object of a subclass of a class
representing the functions internally, which leads us to the conclusion that the
scala source-level closures are translated to regular class objects.

\begin{figure}
\begin{scalaCode}
def map[B, That](f: A => B)(implicit bf: CanBuildFrom[Repr, B, That]): That =
{
    def builder = {
      val b = bf(repr)
      b.sizeHint(this)
      b
    }
    val b = builder
    for (x <- this) b += f(x)
    b.result
}
\end{scalaCode}
\caption[Scala's default \texttt{map} implementation]{Scala's default \texttt{map} implementation}
\label{scala_map}
\end{figure}

In Figure~\ref{scala_map} we can see the Scala standard library's \texttt{map} implementation. Here \texttt{f} is the function object scalac passed during the \texttt{map} call, i.e., the
object resulted from \sv{new \$anonfun()} call from Figure~\ref{scalac_map_internal_repr}. 
\texttt{f(x)} invocation will expand to
\sv{f.apply(x)} and, therefore, it is a normal method call on object \texttt{f}. We can easily see
how similar this \texttt{map} definition is with the one provided in Figure~\ref{java_generic_map}.
They both suffer from The Problem. Here, \texttt{f}'s runtime type will, usually, be
different on each \texttt{map} call since the functions we pass are generally different.
As we explained in Chapter~\ref{introduction}, such calls are called megamorphic virtual calls and, currently, cannot be inlined efficiently
by the JVM. So on each \texttt{map} we
generally have the added overhead of a dynamic call to the passed function
object. Even worse, the lack of knowledge of what goes on inside the virtual
call prevents all kinds of crucial optimizations.

We can see how \textsc{ft-declosurify} translates the respective piece of code,
\sv{List(1, 2, 3).\allowbreak{}macroMap(\_ + 1)}, in Figure~\ref{list_expansion_1}. The only difference here is the use of \texttt{macroMap} instead
of the plain \texttt{map}.

\begin{figure}
\begin{scalaCode}
{
  private def local1(x$1: Int): Int = x$1.+(1);
  private def builder1: scala.collection.mutable.Builder[Int,List[Int]] = {
    val b: scala.collection.mutable.Builder[Int,List[Int]] =
immutable.this.List.canBuildFrom[Int].apply();
    b.sizeHint(immutable.this.List.apply[Int](1, 2, 3));
    b
  };
  val buf = builder1();
  var these = immutable.this.List.apply[Int](1, 2, 3);
  while(!these.isEmpty){
    buf.+=(local1(these.head));
    these = these.tail
  };
  buf.result
}
\end{scalaCode}
\caption[Expanded \sv{List(1, 2, 3).macroMap(\_ + 1)}]{Expanded \sv{List(1, 2, 3).macroMap(\_ + 1)}}
\label{list_expansion_1}
\end{figure}

Here we use a \sv{scala.\allowbreak{}collection.\allowbreak{}mutable.\allowbreak{}Builder} object to construct the target
collection. The builder itself is created from the
\texttt{canBuildFrom} object the compiler passed in implicitly, as we will explain more thoroughly in section~\ref{ft_declosurify_impl}. 
The builder creation and initialization takes place in the
\texttt{builder1} method. While we traverse the prefix, we apply the \texttt{local1} function to
all elements and we append (through method \texttt{+=}) them to the builder. When the
traversal is over, we call the builder's result method which returns the full
target collection we want (a \texttt{List} in this example).

In \texttt{macroMap}'s transformation, we see that there is no explicit call to any 
\texttt{map} method. Instead, the list's traversal happens directly within a while loop 
where the \texttt{local1} local free method is applied to all of the list's elements. 
The \texttt{local1} method is called a free method since it's not directly attached to any class or object.
During scalac's ``flatten'' phase, it will be lifted and become a method of
the enclosing class with a new mangled name. That way \texttt{local1}'s receiver runtime type
will, likely, remain unchanged during a program's execution. Therefore, it's much easier for JVM
to inline \texttt{local1} and reason about further optimizations.

We can easily see that \textsc{ft-declosurify} works well for cases where the closure
is statically available at the call-site, like \sv{List(1,2,3).macroMap(x => x + 1)},
where it will be expanded to something like

\begin{scalaCode}
{
  def local1(x$1: Int): Int = x$1.+(1);
  ...
}
\end{scalaCode}
according to the transformations above. This case doesn't solve The Problem
at its core but, instead, it ``sidesteps'' it at a metalinguistic level, by
eager compile-time inlining and by ``breaking'' and transforming the passed function
object to a local method.

A more interesting case is when we have something like \sv{List(1, 2,
3).macroMap(myf)}, where we don't know much about the \texttt{myf} function except, let's
say, its type is \sv{Int => Int}. Then, \texttt{macroMap} will be expanded into something
like:

\begin{scalaCode}
{
 def local1(x$1: Int): Int = myf.apply(x$1);
  ...
}
\end{scalaCode}

Initially, it seems that we fall back to The Problem again, without having any
advantage, since \sv{myf.apply} seems to be a megamorphic virtual call again.
But, in a program, all \texttt{macroMap} instances will be expanded at the call-site and their \texttt{myf}'s
runtime type will, usually, remain the same during each call and, theoretically,
the type profiler can inline each \sv{myf.apply}, which again alleviates 
The Problem. So, in both cases, we do achieve some wins against The Problem.

In summary, \texttt{macroMap} transformation takes advantage of:
\begin{itemize}
 \item Knowing the static type of \texttt{macroMap}'s receiver, because it can apply different transformations depending on the type.
  \item Knowing the static type of the \texttt{macroMap}'s function, because it can transform it in a local method and, make it inlining friendly.
  \item Having a fixed built-in implementation, applying eager compile-time inlining.
\end{itemize}


\section{\textsc{ft-declosurify} Implementation Specifics}
\label{ft_declosurify_impl}

In this section, we will see the \textsc{ft-declosurify} implementation in more detail.~\footnote{The full implementation source code is available at \url{http://github.com/geo-kollias/scala/tree/ft-declosurify}}

\subsection{ \texttt{macroMap}/\texttt{macroForeach} definitions}

The \texttt{macroForeach} method, just like the standard \texttt{foreach}, is meant to traverse
all elements of the collection, and apply the given operation, \texttt{f}, to each
element. The invocation of \texttt{f} is done for its side effect only; in fact any
function result of \texttt{f} is discarded. Similarly, \texttt{macroMap}, just like the standard
\texttt{map}, traverses all elements of the collection, and applies the given operation,
\texttt{f}, to each element but, also, it always returns a collection whose type depends
on \texttt{f}. We could see \texttt{macroMap}'s functionality as a superset of \texttt{macroForeach}'s
functionality and that's true on the implementation side too. The same code
handles both cases but, in the case of \texttt{macroForeach}, we ignore the result. For
that reason and without any loss of generality we will be referring only to
\texttt{macroMap} in the rest of the text.

\texttt{macroMap} is defined in standard library's \sv{scala.\allowbreak{}collection.\allowbreak{}TraversableLike} trait, where the
default \texttt{map} method is defined too. Its definition is an unusual one:

\begin{scalaCode}
def macroMap[B, That](f0: A => B)(implicit bf: CanBuildFrom[Repr, B, That]): That = ???'
\end{scalaCode}

Instead of a usual implementation on the right hand side, we see a \texttt{???}. The \texttt{???}
is an actual special Scala method which is used to throw \texttt{NotImplementedError}
exceptions. Our \texttt{macroMap} implementation will be generated at compile-time, as
we will explain in the next subsections, so no exception is raised.

The above signature is exactly the same as the \texttt{map}'s one. Regarding the
type parameters, \texttt{A} is the collection's element type and, therefore, \texttt{f0}'s
argument type, \texttt{B} is \texttt{f0}'s return type, \texttt{Repr} is the underlying collection's type
and \texttt{That} is the generated collection's type. Also, as we said, \texttt{macroMap} is a
curried method. The first parameter list accepts the operation's function. The
second list gets an implicit \texttt{CanBuildFrom} object. \texttt{CanBuildFrom} objects are
builder factories which generate the appropriate
\sv{scala.\allowbreak{}collection.\allowbreak{}generic.Builder} objects depending on their type parameters.
Generally, builders makes the creation of a new collection out of existing ones
easier and more maintainable. For example, a \sv{CanBuildFrom[\allowbreak{}List[\allowbreak{}String], Int,
Array[Int]]} object will create a \sv{Builder[Int, Array[Int]]} object which can
create an \texttt{Array} of integers out of a \texttt{List} of strings. Since \texttt{bf} is implicit, it
will usually be generated and passed by the compiler automatically.


\subsection{Linking \texttt{macroMap}/\texttt{macroForeach} definitions with implementations}
\label{ft_decl_std_lib}

The upstream \textsc{declosurify} project can be used as a library adding the -extension-
methods \texttt{macroMap} and \texttt{macroForeach} on all Scala collections that implement the
\sv{scala.\allowbreak{}collection.\allowbreak{}TraversableOnce} trait and \texttt{Array}s, through Scala's implicit
conversions~\cite{oliveira_type_2010}. Since, ultimately, we would like to
substitute the default \texttt{map} and \texttt{foreach} methods with the \texttt{macroMap} and
\texttt{macroForeach} implementations respectively, we had to move the definitions
of \texttt{macroMap}, \texttt{macroForeach} inside the Scala standard library. The Scala standard
library doesn't have access to the new compile-time reflection capabilities
since it doesn't depend on the \sv{scala-reflect} and \sv{scala-compiler} packages, so we
cannot have macro definitions in it. As a solution, we used the scala compiler's
\emph{FastTrack} mechanism defined in the \sv{scala.tools.reflect.FastTrack} trait.

\texttt{FastTrack} is a low level mechanism of scalac to invoke macro
expansions for builtin macros and it is, also, the key machinery used by our
implementation in order to register and invoke our transformations. One builtin
macro that uses this mechanism is the \texttt{reify} function itself that we saw in the
previous chapter. The \texttt{FastTrack} module uses a registry where one links
methods' compiler \texttt{Symbol}s with special handler methods. For example, whenever
the compiler sees a \texttt{Symbol} representing an application of \texttt{reify}, it calls its
respective \texttt{FastTrack} special handler which has access to the reify application's
call-site context and all of  its arguments' ASTs,
through pattern matching. The call-site context object, generally, holds
call-site information like the macro application's enclosing method, its
enclosing class, its line in the file. The ASTs contain the internal
representation of the macro application like \sv{List(1, 2, 3).macroMap(\_ + 1)}
i.e., the receiver object, the \texttt{macroMap} method call and its function argument.
Pattern matching happens upon these ASTs so, theoretically, we can choose to
match successfully against \sv{List(1,2,3).macroMap(\_ + 1)} but not against \sv{val
r = List(1,2,3); r.macroMap(\_ + 1)} depending on our needs. If the pattern
matching is successful, its respective compiler handler will eventually generate
the AST which replaces the macro application's AST at the call-site, completing
that way the low-level macro expansion. In our case, whenever scalac finds a
\texttt{Symbol} of an application of the \sv{scala.\allowbreak{}collection.\allowbreak{}TraversableLike}'s
\texttt{macroMap} method, it triggers the expansion we will describe in the next
subsection.

Regular Scala macros implementations can be generic in the sense that they can
be customized with type parameters as any regular Scala method. On top of that,
the Scala compiler allows us to ``tag'' each of these type parameters with
special compiler-generated objects, of type (\texttt{Weak})\texttt{TypeTag}~\cite{oliveira_type_2010}, that store the
type parameters' full types on each call and make them available at runtime. In
short, it's Scala's solution against JVM's type erasure. This machinery is
especially useful for macros, since we can inspect these \texttt{TypeTag} objects and
generate the most suitable code for each occasion. Unfortunately, using the low
level compiler's \texttt{FastTrack} mechanism doesn't permit us using this facility.
Instead, we are forced to work directly with the compiler's internal type
representations lowering the level of abstraction we can work with.

\texttt{FastTrack}'s special handler for the \texttt{macroMap} method, after
doing some preprocessing on the pattern matched ASTs, will call the method
that does the actual transformation and code generation. 


\subsection{Transformation Method Interface}

Firstly, let's see the transformation method's signature and what arguments it
obtains from the handler. This method is defined in
\sv{scala.\allowbreak{}tools.\allowbreak{}reflect.\allowbreak{}declosurify.\allowbreak{}Declosurify} object 

\begin{scalaCode}
def mapInfix[A, B](c0: Ctx)(f0: c0.Expr[A => B], inElemTpe: c0.Type, outElemTpe: c0.Type, inCollTpe: c0.Type, outCollTpe: c0.Type, bfTree: c0.Tree): c0.Tree
\end{scalaCode}

Despite its name,
\texttt{mapInfix} can generate ASTs for both \texttt{macroMap} and \texttt{macroForeach}. It's a curried
method parameterized on the received function's input (\texttt{A}) and return (\texttt{B}) types.
In the first parameter list, \texttt{c0} is a context object as we described it in subsection~\ref{ft_decl_std_lib},
giving us access to the macro application's call-site information. The most
important field of \texttt{c0} is the \emph{prefix} which represents the receiver collection
object of the \texttt{macroMap} application. We will call it simply prefix
from now on. In the second parameter
list, \texttt{f0} represents the function's AST which is going to be applied to the
prefix. Its type, \texttt{Expr}, wraps an AST and an internal type tag
(\texttt{TypeTag}) to provide access to the type of the tree. As we mentioned
before, using \texttt{Expr} doesn't really help us working at that level of the
compiler, since we cannot really exploit the (\texttt{Weak})\texttt{TypeTag}'s facilities. The
next five arguments are:
\begin{itemize}
 \item
  \texttt{inElemTpe}: the internal compiler type of prefix's elements
 \item
  \texttt{outElemTpe}: the internal compiler type of the elements of the collection our
\texttt{macroMap} is going to return
 \item
  \texttt{inCollTpe}: the internal compiler type of the prefix
 \item
  \texttt{outCollTpe}: the internal compiler type of the collection our \texttt{macroMap} is going
to return
 \item
  \texttt{bfTree}: the AST of the implicit \sv{scala.\allowbreak{}collection.\allowbreak{}generic.CanBuildFrom} object that got
inserted by the compiler at the \texttt{macroMap} call-site. This provides us with an
easy and accurate way to create the appropriate builder for the generated
collection, taking full advantage of all the existing implicit \texttt{CanBuildFrom} objects
that are declared in the standard library~\cite{moors2009type} .
\end{itemize}

It's easy to observe that all of second parameter list types are prepended
with the context object \texttt{c0}. This qualified notation realizes the notion of
Scala's path-dependent types. For example, here we choose the \texttt{Expr} that is
defined inside the \texttt{c0} object. Generally, if we have two different objects \texttt{c0} and
\texttt{c1} of the same type which include an inner type \texttt{MyType}, e.g., through type
member or inner class,  then \sv{c0.MyType} is a different type from \sv{c1.MyType} in
Scala. The same holds for the \sv{c0.Type}s that follow.


\subsection{Transformation Requirements}

Now that we know what our transformation method can operate upon, we can examine the
main points of the actual transformation. One of the first things \textsc{ft-declosurify}
checks is if the \texttt{outCollTpe} is \texttt{Unit} type. If it's \texttt{Unit} then
it means the macro is applied only for its side-effects so it's a \texttt{foreach} call.
Instead of relying on this heuristic, newer version of the Scala macros provide
us with the exact name of the calling method.

After that, we check three conditions to ensure that a typical
transformation can take place:
\begin{itemize}
 \item
  the \texttt{f0} AST is actually a function AST or a block whose last
expression is a function, since, in Scala, the value of a block is the block's
last expression.
 \item
    the \texttt{f0} AST doesn't contain any \texttt{return} expressions.
 \item
    macro application is enclosed in a method. Currently, that limitation makes
the transformation easier.
\end{itemize}

If any of the above points is not satisfied, then the mapInfix falls back to the
default \texttt{map}/\texttt{foreach} implementation by generating an AST which
calls the \texttt{map}/\texttt{foreach} method on the same prefix object:

\begin{scalaCode}
def mkFallbackImpl: c.Tree = {
  val name: TermName = if (isForeach) "foreach" else "map"
  val pre = c.prefix.tree
  val fallbacktree = Apply(Select(pre, name), f0.tree :: Nil)
  fallbacktree
}
\end{scalaCode}


The next important step is the transformation of the passed closure AST
into a local free method AST:

\begin{scalaCode}
def functionToLocalMethod(fnTree: Function): DefDef = {
  val Function(fparams, fbody) = fnTree
  val frestpe = fbody.tpe
  val fsyms   = fparams map (_.symbol)
  val vparams = for (vd @ ValDef(mods, name, tpt, _) <- fparams) yield ValDef(mods, name, TypeTree(vd.symbol.typeSignature.normalize), EmptyTree)
  val method  = newLocalMethod(freshName("local"), vparams, frestpe)
  val tree    = DefDef(NoMods, freshName("local"), Nil, List(vparams), TypeTree(frestpe), c.resetAllAttrs(fbody.duplicate))

  tree setSymbol method
  c.resetAllAttrs(tree)
  c.typeCheck(tree).asInstanceOf[DefDef]
}
\end{scalaCode}

For example, the passed closure \sv{x => x + 1} would be transformed to:
\begin{scalaCode}
def local1(x$1: Int): Int = x$1.+(1);
\end{scalaCode}
A closure like \sv{x => x + y} where \texttt{y} is defined in the local scope would be
transformed to something like:
\begin{scalaCode}
def local1(x: Int): Int = x.+(TestMacroMapObject.this.y)
\end{scalaCode}
Also, a closure like \sv{\{println("hi"); x => x + 1\}} would be transformed to:
\begin{scalaCode}
println("hi");
def local1(x: Int): Int = x.+(1);
\end{scalaCode}

\subsection{Transformation Choice and Idiosyncrasies}

The transformation choice depends, primarily, upon the prefix's static type. Right now,
there are three different transformation paths for different kinds of
prefixes. We will check them in the following subsections, but all of them generate
similar code with the original Scala \texttt{map} implementation, although they use 
more low-level constructs.

The reasons we need different implementations for different kinds of
prefixes and target collections are:

\begin{itemize}
 \item
  Each kind has different API, e.g., different supported methods.
 \item
  Different implementation logic is needed for each kind in order to produce
faster code, by exploiting each kinds' specific characteristics since the same
methods might take different time on different collections.
\end{itemize}

For example, trait \texttt{Seq} has two subtraits \texttt{LinearSeq} and \texttt{IndexedSeq}.
These do not add any new operations, but each offers different performance characteristics: A
linear sequence has efficient \texttt{head} and \texttt{tail} operations, whereas an indexed
sequence has efficient \texttt{apply}, \texttt{length}, and (if mutable) \texttt{update} operations. 

Also, we will see how important reifying and splicing operations are. The
generated ASTs we splice inside the \texttt{reify} expression are constructed with one of
these three ways depending on the occasion:
\begin{itemize}
 \item
  we get them directly from the \texttt{FastTrack}'s pattern matching.
 \item
  we get them by using the \texttt{Symbol} and \texttt{Type} APIs.
 \item
  we construct them manually.
\end{itemize}

Whatever AST is returned from the \texttt{reify} will eventually replace the macro
application's AST in the first place.

Finally, we should also mention that in contrast to most of the other collections operations, 
a \texttt{map} operation can generate a collection that has the same type constructor as the prefix
collection but possibly with a different element type, or even a entirely
different collection type. As an example of the former case, if \texttt{f} is a function
from \texttt{String} to \texttt{Int}, and \texttt{xs} is a \sv{List[String]}, then \sv{xs map f} gives a \sv{List[Int]}.
Likewise, if \texttt{ys} is an \sv{Array[String]}, then \sv{ys map f} gives a \sv{Array[Int]}. As an
example of the later case,
\begin{scalaCode}
 Map("a" -> 1, "b" -> 2) map { case (x, y) => y}
\end{scalaCode}
returns \sv{List(1, 2)} of type \sv{scala.\allowbreak{}collection.\allowbreak{}immutable.\allowbreak{}Iterable[Int]},
which is different from the prefix's \texttt{Map} type. The upstream declosurify project
could handle partially only the first case. \textsc{ft-declosurify} can handle both of
the cases successfully due to the introduction of the implicit \texttt{CanBuildFrom}
object, making it a more general purpose transformation tool.


\subsection{\texttt{Array}, \texttt{scala.collection.mutable.ArraysOps}  and\\
\texttt{scala.\allowbreak{}collection.\allowbreak{}mutable.\allowbreak{}IndexedSeq} Transformation}

\sv{scala.Array} and all collections that implement the 
\sv{scala.\allowbreak{}collection.\allowbreak{}mutable.\allowbreak{}ArrayOps} and \sv{scala.\allowbreak{}collection.\allowbreak{}mutable.\allowbreak{}IndexedSeq}
traits, like \sv{scala.\allowbreak{}collection.\allowbreak{}mutable.\allowbreak{}ArraySeq},
 \sv{scala.\allowbreak{}collection.\allowbreak{}mutable.\allowbreak{}StringBuilder}, 
\sv{scala.\allowbreak{}collection.\allowbreak{}mutable.\allowbreak{}ArrayBuffer},
share the same transformation, since all of them are \emph{mutable indexed
sequences}. The transformation code is in Figure~\ref{mut_ind_transf}.

\begin{figure}
\begin{scalaCode}
def mkMutIndexed[Prefix](prefixTree: Tree): c.Tree = {
  val prefix   = c.Expr[Prefix](prefixTree)
  val len      = c.Expr[Int]('xs dot 'length) // might be array or indexedseq
  val call     = c.Expr[Unit](('buf dot 'update)('i, closure('xs('i))))
  def mkResult = c.Expr[Nothing](if (isForeach) mkUnit else 'buf)

  val arrCons  =   Apply(Select(New(TypeTree(outCollTpe)), nme.CONSTRUCTOR), List(('xs dot 'length).lhs))
  val builderVal = c.Expr[Prefix](arrCons)
  
  reify {
    closureDef.splice
    val xs = prefix.splice
    var buf = builderVal.splice
    var i  = 0
    while (i < len.splice) {
      call.splice
      i += 1
    }
    mkResult.splice
  }.tree
}
\end{scalaCode}
\caption[Mutable indexed sequences transformation code]{Mutable indexed sequences transformation code}
\label{mut_ind_transf}
\end{figure}

As an example, any \texttt{macroMap} applications on \texttt{Array}s like:
\begin{scalaCode}
Array(1, 2, 3).macroMap(_ + 1)
\end{scalaCode}
will expand to Figure~\ref{array_expansion}'s code.

\begin{figure}
\begin{scalaCode}
{
  def local1(x$1: Int): Int = x$1.+(1);
  val xs: Array[Int] = scala.this.Predef.intArrayOps(scala.Array.apply(1,
scala.this.Predef.wrapIntArray(Array[Int]{2, 3}))).repr();
  var buf: Array[Int] = new Array[Int](xs.length());
  var i: Int = 0;
  while(i.<(xs.length())){
    buf.update(i, local1(xs.apply(i))); // buf(i) = (local1(r(i)));
    i = i.+(1)
  };
  buf
}
\end{scalaCode}
\caption[Expanded \sv{Array(1, 2, 3).macroMap(\_ + 1)}]{Expanded \sv{Array(1, 2, 3).macroMap(\_ + 1)}}
\label{array_expansion}
\end{figure}

Exactly the same transformation applies for all the other collections that fall
into this category. The reasons we chose this transformation path for this
category are:

\begin{itemize}
 \item 
  only this category's collections accept a length in the constructor
 \item
  the length method here takes constant time
 \item
  the element selection through xs.apply(i) takes constant time
 \item
  the update method is supported, since all collections of this category are
mutable, and takes constant time
\end{itemize}

Also, all of this category's collections can only be mutated through the
\texttt{update} method (the assignment operator) on a specific index, with one
exception, the \sv{scala.\allowbreak{}collection.\allowbreak{}mutable.\allowbreak{}ArrayBuffer}. 
\texttt{ArrayBuffer} supports mutating methods for appending/removing elements which could affect the
semantics of \texttt{macroMap} in case we mutate the underlying collection in the
function we pass in \texttt{macroMap}. Our transformation seems to behave in the same
way as the default \texttt{map} method. For example both
\begin{scalaCode}
buf map {x => buf += x; x+1 }
\end{scalaCode}
and
\begin{scalaCode}
buf macroMap { x => buf += x; x+1 }
\end{scalaCode}
return \sv{ArrayBuffer(2, 3, 4)}.
So, we assume the \texttt{macroMap} keeps the
same semantics on the \texttt{ArrayBuffer} as the original \texttt{map} method.

In the next chapter will see how much faster this version is compared to the
default \texttt{map} on \texttt{Array}s and \texttt{ArraySeq}s.

\subsection{\texttt{scala.collection.LinearSeq} Transformation}

This category includes \sloppy{\sv{scala.\allowbreak{}collection.\allowbreak{}}\{\sv{\allowbreak{}immutable.\allowbreak{}List},
\sv{\allowbreak{}immutable.\allowbreak{}Stream}, \sv{\allowbreak{}immutable.\allowbreak{}Queue},
\sv{\allowbreak{}immutable.\allowbreak{}Stack}, \sv{\allowbreak{}mutable.\allowbreak{}MutableList},
\sv{\allowbreak{}mutable.\allowbreak{}MutableList}, \sv{\allowbreak{}mutable.\allowbreak{}LinkedList},
\sv{\allowbreak{}mutable.\allowbreak{}DoubleLinkedList}\}}. All of these collections are \emph{linear}.
For example, in this category's transformation we couldn't use the \texttt{apply} (for
indexing) or the \texttt{length} method because both of them take linear time, instead of
constant. The transformation is in Figure~\ref{lin_transf}.

\begin{figure}
\begin{scalaCode}
 def mkLinear(prefixTree: Tree): c.Tree = {
  val prefix = c.Expr[Lin[A]](prefixTree)
  val call   = mkCall('these dot 'head)

  reify {
    closureDef.splice
    builderDef.splice
    builderVal.splice
    var these = prefix.splice
    while (!these.isEmpty) {
      call.splice
      these = these.tail
    }
    mkResult.splice
  }.tree
}
\end{scalaCode}
\caption[Linear sequences transformation code]{Linear sequences transformation code}
\label{lin_transf}
\end{figure}


As an example, any
\texttt{macroMap} applications on \sv{scala.\allowbreak{}collection.\allowbreak{}immutable.\allowbreak{}List} like:
\begin{scalaCode}
List(1, 2, 3).macroMap(_ + 1)
\end{scalaCode}
will be replaced from Figure's~\ref{list_expansion}'s code.

\begin{figure}
\begin{scalaCode}
{
  private def local1(x$1: Int): Int = x$1.+(1);
  private def builder1: scala.collection.mutable.Builder[Int,List[Int]] = {
    val b: scala.collection.mutable.Builder[Int,List[Int]] =
immutable.this.List.canBuildFrom[Int].apply();
    b.sizeHint(immutable.this.List.apply[Int](1, 2, 3));
    b
  };
  val buf = builder1();
  var these = immutable.this.List.apply[Int](1, 2, 3);
  while(!these.isEmpty){
    buf.+=(local1(these.head));
    these = these.tail
  };
  buf.result
}
\end{scalaCode}
\caption[Expanded \sv{List(1, 2, 3).macroMap(\_ + 1)}]{Expanded \sv{List(1, 2, 3).macroMap(\_ + 1)}}
\label{list_expansion}
\end{figure}

This category uses this transformation because:
\begin{itemize}
 \item
  \texttt{isEmpty} takes constant time
 \item
  \texttt{head} takes constant time
 \item
  \texttt{tail} takes constant time
\end{itemize}

In the next chapter will see how much faster this version is compared to the
default \texttt{map} on \texttt{List}s.

\subsection{\texttt{scala.collection.Traversable} Transformation}

This category includes all collections that don't fit in any of the previous
categories. Roughly, it includes all kinds of \texttt{Set}s, \texttt{Map}s, \texttt{Buffer}s and
\sv{immutable.IndexedSeq}s. Obviously, since it includes such a broad range of
collections we are allowed to use only methods that are supported from all the
collections, meaning that we cannot use the \texttt{length}, \texttt{apply}, \texttt{head}, \texttt{tail} methods of the
previous transformations. The transformation is in Figure~\ref{trav_transf}.

\begin{figure}
\begin{scalaCode}
 def mkTraversable(prefixTree: Tree): c.Tree = {
  val prefix = c.Expr[Traversable[A]](prefixTree)
  val call   = mkCall('it dot 'next)

  reify {
    closureDef.splice
    builderDef.splice
    builderVal.splice
    val it = prefix.splice.toIterator
    while (it.hasNext)
      call.splice

    mkResult.splice
  }.tree
}
\end{scalaCode}
\caption[Traversable sequences transformation code]{Traversable sequences transformation code}
\label{trav_transf}
\end{figure}

As an
example, any \texttt{macroMap} applications on \sv{scala.\allowbreak{}collection.\allowbreak{}immutable.\allowbreak{}Set} like:

\begin{scalaCode}
Set(1, 2, 3).macroMap(_ + 1)
\end{scalaCode}
will expand to Figure~\ref{set_expansion}'s code.

\begin{figure}
\begin{scalaCode}
{
  private def local1(x$1: Int): Int = x$1.+(1);
  private def builder1:
scala.collection.mutable.Builder[Int,scala.collection.immutable.Set[Int]] = {
    val b:
scala.collection.mutable.Builder[Int,scala.collection.immutable.Set[Int]] =
immutable.this.Set.canBuildFrom[Int].apply();
    b.sizeHint(r);
    b
  };
  val buf = builder1();
  val it = r.toIterator;
  while(it.hasNext){
    buf.+=(local1(it.next));
  };
  buf.result
}
\end{scalaCode}
\caption[Expanded \sv{Set(1, 2, 3).macroMap(\_ + 1)}]{Expanded \sv{Set(1, 2, 3).macroMap(\_ + 1)}}
\label{set_expansion}
\end{figure}

Here, we use the \texttt{toIterator} method to generate a prefix's iterator object. All
collections have an \texttt{toIterator} method since all of them implement the
\sv{scala.\allowbreak{}collection.\allowbreak{}Iterable} trait. Each collection implements it using its
\sv{scala.\allowbreak{}collection.\allowbreak{}Traversable}'s \texttt{foreach} method, which is implemented according to
each collection's special characteristics in order to be more efficient.

For the collections that implement the \sv{immutable.IndexedSeq} trait, like \texttt{Vector},
we could have used a modified version of the mutable indexed sequences
transformation where we would use a builder object to build the target
collection, since we cannot mutate it in place. Interestingly, our experiments
showed us that solution was slower than the current one.