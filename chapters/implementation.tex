This chapter presents the core of our work. Section XX will give us a
thorough overview of ft-declosurify and section XX will describe the major
ft-declosurify implementation details.

Section: ft-declosurify Overview

For understanding at a high-level what ft-declosurify does let's see how an
example expression List(1, 2, 3).map(_ + 1) is translated from both the scalac
and the ft-declosurify points of view. List(1, 2, 3).map(_ + 1) is a shortcut
for List(1, 2, 3).map(x => x + 1), where x => x + 1 is an anonymous function
-closure- that returns its argument incremented by one.
Applying that function to the List(1, 2, 3) will result in a new list List(2,
3, 4). Since Scala code is translated into Java bytecode at the end and since
Java  doesn't support any notion of functions inherently, this anonymous
function should be translated somehow in constructs that are supported by the
Java bytecode. The trivial Scala program below:

package src.main.scala
object ListMapCompilation extends App {
  def f = List(1, 2, 3).map(_ + 1)
}

is translated internally by scalac to:

package src.main.scala {
  object ListMapCompilation extends Object with App {
    def <init>(): src.main.scala.ListMapCompilation.type = {
      ListMapCompilation.super.<init>();
      ()
    };
    def f(): List[Int] = immutable.this.List.apply[Int](Array[Int]{1, 2,
3}).map[Int, List[Int]]({
      @SerialVersionUID(0) final <synthetic> class $anonfun extends
scala.runtime.AbstractFunction1[Int,Int] with Serializable {
        def <init>(): anonymous class $anonfun = {
          $anonfun.super.<init>();
          ()
        };
        final def apply(x$1: Int): Int = x$1.+(1)
      };
      (new anonymous class $anonfun(): Int => Int)
    }, immutable.this.List.canBuildFrom[Int]())
  }
}

The code listing above as well as most of the following listings are slightly
abbreviated for readability reasons.
We can see that scalac converts the _ + 1 function into a block of code
(piece of code between two braces), where a class, called $anonfun, is defined.
That class extends the AbstractFunction1[Int,Int] class, which is an abstract
class that represents the functions that take one integer argument and return
another integer. Inside the class an apply method is defined which is called
whenever we apply a class object to one integer argument. The apply body
returns  its argument incremented by one,  e.g., val a1 = new $anonfun();
a1(5); returns 6. Just after the class definition, scalac creates a new object
of this class and this is what is actually returned from that block of code.
Eventually, the _ + 1 is substituted by an object of a subclass of a class
representing functions internally, which leads us to the conclusion that the
scala source-level closures are translated to function class objects. Below we
attach the basic map implementation in the standard Scala library:

def map[B, That](f: A => B)(implicit bf: CanBuildFrom[Repr, B, That]): That =
{
    def builder = {
      val b = bf(repr)
      b.sizeHint(this)
      b
    }
    val b = builder
    for (x <- this) b += f(x)
    b.result
}

Here f is the function object scalac passed during the map call, i.e., the
object resulted from new $anonfun() above. f(x) invocation will be expanded to
f.apply(x) and it is a normal method call on object f. We can easily see
how similar this map definition is with the one provided in Listing XX.
They both suffer from The Problem. Here, f's runtime type will, usually, be
different on each map call since the functions we pass are generally different.
Such calls are called mega-morphic virtual calls and, currently, are not inlined
by the JVM as we exlained in the introduction. So on each map we
generally have the added overhead of a dynamic call to the passed function
object. Even worse, the lack of knowledge of what goes on inside the virtual
call prevents all kinds of crucial optimizations.

Let's see how ft-declosurify translates a very similar piece of code:

package src.main.scala
object ListMapCompilation extends App {
  def f = List(1, 2, 3).macroMap(_ + 1)
}

The only difference here is the use of macroMap instead of the plain map. It's
translated into:

[TODO: obsolete example, cbf & builder free method have been added]
[TODO: integrate builder description to ``ft-declosurify big picture`` section]
Here we use a scala.collection.mutable.Builder object to construct the target
collection, which is a Vector too. The builder itself is created from the
canBuildFrom object the compiler passed in implicitly as we explained
previously. The builder creation and initialization takes place in the
builder1 method. While we traverse the prefix, we apply the local function to
all elements and we append (through method +=) them to the builder. When the
traversal is over, we call the builder's result method which returns the full
target collection we want (a Vector in this example).
package src.main.scala {
  object ListMapCompilation extends Object with App {
    def <init>(): src.main.scala.ListMapCompilation.type = {
      ListMapCompilation.super.<init>();
      ()
    };
    def f(): List[Int] = {
      private def local1(x$1: Int): Int = x$1.+(1);
      val buf: scala.collection.mutable.Builder[Int,List[Int]] =
newBuilder[Int]();
      var these: List[Int] =
immutable.this.List.apply[Int](scala.this.Predef.wrapIntArray(Array[Int]{1, 2,
3}));
      while(these.isEmpty().unary_!()){
        buf.+=(local1(these.head()));
        these = these.tail()
      };
      buf.result()
    }
  }
}

Here we see that there is no explicit call to any map method. Instead the
list's traveral happens directly within a while loop where the local1 local free
method is applied to all of the list's elements. The local1 method is called a
free method since it's not directly attached to any class or object.
During scalac's ``flatten'' phase, it will be lifted and become a method of
ListMapCompilation with a new mangled name. In particular the code
above will become:

package src.main.scala {
  object ListMapCompilation extends Object with App {
    def f(): List = {
      val buf: collection.mutable.Builder =
scala.collection.immutable.List.newBuilder();
      var these: List =
immutable.this.List.apply(scala.this.Predef.wrapIntArray(Array[Int]{1, 2, 3}));
     
      while(these.isEmpty().unary_!()){
        buf.+=(scala.Int.box(ListMapCompilation.this.local1\$1(
scala.Int.unbox(these.head()))));
        these = these.tail().$asInstanceOf[List]()
      };
      buf.result().$asInstanceOf[List]()
    };
    final private[this] def local1$1(x$1: Int): Int = x$1.+(1);
    def <init>(): src.main.scala.ListMapCompilation.type = {
      ListMapCompilation.super.<init>();
      ListMapCompilation.this.$asInstanceOf[App$class]()./*App$class*/$init$();
      ()
    }
  }
}

As we can see now, local1is lifted to local1\$1 and its receiver becomes
ListMapCompilation which will remain unchanged during the program execution.
It's much easier for JVM to inline the local1$1 method and reason about further
optimizations.

We can easily see that ft-declosurify works well for cases where the closure
is statically available at the call-site like List(1,2,3).macroMap(x => x + 1),
where it will be expanded to something like
{
  ...
  def local1$1(x$1: Int): Int = x$1.+(1);
  ...
} 
according to the transformations above. This case doesn't solve The Problem
at its core but, instead, it "sidesteps" it at a "metalinguistic" level, by
eager compile-time inlining.

A more interesting case is when we have something like List(1, 2,
3).macroMap(myf), where we don't know much about the myf function except, let's
say, its type is Int => Int. Then, macroMap will be expanded to something
like 
{
  ...
 def local1$1(x$1: Int): Int = myf.apply(x$1);
  ...
}.
Initally, it seems that we fall back to The Problem without having any
advantage. But since all macroMaps will be expanded at the call-site, myf's
runtime type will remain the same during each macroMap call and, theoretically,
the type profiler can inline each macroMap's myf.apply -provided that it's less
than 35 bytes-, which again alleviates The Problem.

So in both cases, we do achieve some wins against The Problem.


Section: ft-declosurify Implementation Specifics

Subsection: macroMap/macroForeach definition

The macroForeach method, just like the standard foreach, is meant to traverse
all elements of the collection, and apply the given operation, f, to each
element. The invocation of f is done for its side effect only; in fact any
function result of f is discarded. Similarly, macroMap, just like the standard
map, traverses all elements of the collection, and applies the given operation,
f, to each element but, also, it always returns a collection whose type depends
on f. We could see macroMap's functionality as a superset of macroForeach's
functionality and that's true on the implementation side too. The same code
handles both cases but, in the case of macroForeach, we ignore the result. For
that reason and without any loss of generality we will be referring only to
macroMap in the rest of the text.

macroMap is defined in scala.collection.TraversableLike trait, where the
standard library's map method is defined too. Its definition is an unusual one:

``def macroMap[B, That](f0: A => B)(implicit bf: CanBuildFrom[Repr, B, That]):
That = ???''

Instead of a usual implementation on the right hand side, we see a ???. The ???
is an actual special Scala method which is used to throw NotImplementedError
exceptions. Our macroMap implementation will be generated at compile-time, as
we will explain in the next subsections, so no exception is raised. It's here
just to prevent syntax errors. 

The above definition signature is exactly the same as the map one. Regarding
type parameters, A is the collection's element type and, therefore, f0's
argument type, B is f0's return type, Repr is the underlying collection's type
and That is the generated collection's type. Also, as we said, macroMap is a
curried method. The first parameter list accepts the operation's function. The
second parameter gets an implicit CanBuildFrom object. CanBuildFrom objects are
builder factories which generate the appropriate
scala.collection.generic.Builder objects depeding on their type parameters.
Generally, builders makes the creation of a new collection out of existing ones
easier and more maintable. For example, a CanBuildFrom[List[String], Int,
Array[Int]] object will create a Builder[Int, Array[Int]] object which can
create an Array of integers out of a List of strings. Since bf is implicit, it
will usually be generated and passed by the compiler automatically.


Subsection: ft-declosurify in the Scala standard library

The upstream declosurify project can be used as a library adding the -extension-
methods macroMap and macroForeach on all Scala collections that implement the
scala.collection.TraversableOnce trait and Arrays, through Scala's implicit
conversions {cite implicit conversions}. Since, ultimately, we would like to
substitute the default map and foreach methods with the macroMap and
macroForeach implementations, respectively, we had to move the definitions
of macroMap, macroForeach inside the Scala standard library. The Scala standard
library doesn't have access to the new compile-time reflection capabilities
since it doesn't depend on the scala-reflect and scala-compiler packages, so we
cannot have macro definitions in it. As a solution, we used the scala compiler's
FastTrack mechanism defined in the scala.tools.reflect.FastTrack trait.

FastTrack is a low level mechanism of scalac to invoke macro
expansions for builtin macros and it is, also, the key machinery used by our
implementation in order to register and invoke our transformations. One builtin
macro that uses this mechanism is the reify function itself that we saw in the
previous chapter. The FastTrack module uses a registry where one links a
method's compiler Symbols with special handler methods. For example, whenever
the compiler sees a Symbol representing an application of reify, it calls its
respective FastTrack special handler which has access to the reify appication's
call-site context and all of  its arguments' Abstract Syntax Trees (ASTs),
through pattern matching. The call-site context object, generally, holds
call-site information like the macro application's enclosing method, its
enclosing class, its line in the file. The ASTs contain the internal
representation of the macro application like ``List(1, 2, 3).macroMap(_ + 1)''
i.e., the receiver object, the macroMap method call and its function argument.
Pattern matching happens upon these ASTs so, theoretically, we can choose to
match successfully against ``List(1,2,3).macroMap(_ + 1)'' but not against ``val
r = List(1,2,3); r.macroMap(_ + 1)'' depending on our needs. If the pattern
matching is successful, its respective compiler handler will eventually generate
the AST which replaces the macro application's AST in the call-site, completing
that way the low-level macro expansion. In our case, whenever scalac finds a
Symbol of an application of the scala.collection.TraversableLike trait's
macroMap method, it triggers the expansion we will describe in the next
subsection.

Regular Scala macros implementations can be generic in the sense that they can
be customized with type parameters as any regular Scala method. On top of that,
the Scala compiler allows us to ``tag'' each of these type parameters with
special compiler-generated objects, of type (Weak)TypeTag [cite], that store the
type parameters' full types on each call and make them available at runtime. In
short, it's Scala's solution against JVM's type erasure. This machinery is
especially useful for macros, since we can inspect these TypeTag objects and
generate the most suitable code for each occasion. Unfortunately, using the low
level compiler's FastTrack mechanism doesn't permit using this facility.
Instead, we are forced to work directly with the compiler's internal type
representations lowering the level of abstraction we can work with.

FastTrack's special handler for the macroMap method, after
doing some preprocessing on the pattern matched ASTs, will call the metod
that does the actual transformation we saw in the previous subsection. 


Subsection: Transformation Method Interface

Firstly, let's see the transformation method's signature and what arguments it
obtains from the handler. This method is defined in
scala.tools.reflect.declosurify.Declosurify object ``def mapInfix[A, B](c0:
Ctx)(f0: c0.Expr[A => B], inElemTpe: c0.Type, outElemTpe: c0.Type, inCollTpe:
c0.Type, outCollTpe: c0.Type, bfTree: c0.Tree): c0.Tree''. Despite its name,
mapInfix can generate ASTs for both macroMap and macroForeach. It's a curried
method parameterized on the received function's input (A) and return (B) types.
In the first parameter list, c0 is a context object as we described it above,
giving us access to the macro application's call-site information. The most
important field of c0 is the ``prefix'' which represents the receiver collection
object of the macroMap application. We will call it simply ``prefix''
from now on. In the second parameter
list, f0 represents the function's AST which is going to be applied to the
prefix. Its type ``Expr'' wraps an AST and an internal type tag
(``TypeTag'') to provide access to the type of the tree. As we mentioned
before, using ``Expr'' doesn't really help us working at that level of the
compiler, since we cannot really exploit the (Weak)TypeTag's facilities. The
next four arguments are:
 - inElemTpe: the internal compiler type of the prefix's elements
 - outElemTpe: the internal compiler type of the elements of the collection our
macroMap
is going to return
- inCollTpe: the internal compiler type of the prefix
- outCollTpe: the internal compiler type of the collection our macroMap is going
to return

The last element is the AST of the implicit
scala.collection.generic.CanBuildFrom [cite adrian moors phd] object that got
inserted by the compiler at the macroMap call-site. This provides us with an
easy and accurate way to create the appropriate builder for the generated
collection taking advantage of all the existing implicit CanBuildFrom objects
that are declared in the standard library.

It's easy to observe that all of second parameter list types are prepended
with the context object c0. This qualified notation realizes the notion of
Scala's path-depedent types. For example, here we choose the ``Expr'' that is
defined inside the c0 object. Generally, if we have two different objects c0 and
c1 of the same type which include an inner type MyType, e.g., through type
member or inner class,  then c0.MyType is a different type from c1.MyType in
Scala. The same holds for the ``c0.Type''s that follow.


Subsection: Transformation Requirements

Knowing what our transformation method can operate upon we can examine the
main points of the actual transformation. One of the first things ft-declosurify
checks is if the ``outCollTpe'' is Unit or not (lines XX-YY). If it's Unit then
it means the macro is applied only for its side-effects so it's a foreach call.
Instead of relying on this heuristic, newer version of the Scala macros provide
us with the exact name of the calling method.

After that, we check three conditions to ensure that a typical
transformation can take place (lines XX-YY):
- the f0 AST is actually a function AST or a block whose last
expression is a function, since, in Scala, the value of a block is the block's
last expression.
- the f0 AST doesn't contain any return expressions.
- macro application is enclosed in a method. Currently, that limitation makes
the transformation easier.

If any of the above points is not satisfied, then the mapInfix falls back to the
default map/foreach implementation by generating an AST which
calls the map/foreach method on the same prefix object (lines XX-YY).

The next important step is the transformation of the passed closure AST
into a local free method AST (lines XX-YY). For example, the passed closure ``x
=> x + 1'' would be transformed to ``def local1(x$1: Int): Int = x$1.+(1);''. A
closure like ``x => x + y'' where y is defined in the local scope would be
transformed to something like ``def local1(x: Int): Int =
x.+(TestMacroMapObject.this.y)''. Also, a closure like ``{println(''hi``); x =>
x + 1}'' would be transformed to ``def local1(x: Int): Int = {println(''hi``);
x.+(1)}'

Subsection: Transformation Choice and Pattern

The transformation choice depends upon both the prefix's type and  the
target collection the CanBuildFrom object has decided to create, which depends
itself on which implicit collections' CanBuildFrom object has been passed to
macroMap from the Scala compiler in an earlier phase.

Right now, there are four different transformation paths for different kinds of
prefixes. We will check them in the following subsections but all of them are
based on the same core transformation pattern. In pseudocode that structure
could be described as:

reify {
  val local_method = local_method_AST.splice // we saw it in the previous
subsection
  val inputCollection = prefix_AST.splice
  var outputCollectionBuf = output_collection_buffer_AST.splice // will
hold or generate the new collection
  while (prefix.hasNext()) {
    val elem = prefix.getNext()
    val result = local_method(elem)
    outputCollectionBuf.add(result)
  }
  outputCollectionBuf.generate()
}

The reasons we need differrent implementations for different kinds of
prefixes and target collections are:
- Each kind has different API, e.g., different supported methods
- Different implementation logic is needed for each kind in order to produce
faster code, by exploiting each kinds' specific characteristics since the same
methods might take different time on different collections.

Since we are more interested in the sequences collections, as they are more
common, trait Seq has two subtraits LinearSeq, and IndexedSeq. These do not add
any new operations, but each offers different performance characteristics: A
linear sequence has efficient head and tail operations, whereas an indexed
sequence has efficient apply, length, and (if mutable) update operations. 

Here, also, we see how important reifying and splicing operations are. The
mentioned ASTs that we splice inside the reify above are constructed with one of
these three ways depending on the occasion:
- we get them directly from the FastTrack's pattern matching
- we get them by using the ``Symbol''s and ``Type'''s APIs
- we construct them manually.

Whatever AST is returned from the reify will eventually replace the macro
application's AST in the first place.


Subsection: Handling map's idiosyncrasies

In contrast to most of the other collections operations, a map operation can
generate a collection that has the same type constructor as the prefix
collection but possibly with a different element type, or even a entirely
different collection type. As an example of the former case, if f is a function
from String to Int, and xs is a List[String], then xs map f gives a List[Int].
Likewise, if ys is an Array[String], then ys map f gives a Array[Int]. As an
example of the later case,   ``Map("a" -> 1, "b" -> 2) map { case (x, y) => y
}'' returns List(1, 2) of type scala.collection.immutable.Iterable[Int],
which is different from the prefix's Map type. The upstream declosurify project
could handle partially only the first case. ft-declosurify can handle both of
the cases successfully due to the introduction of the implicit CanBuildFrom
object, making it more general purpose.


Subsection: Array, scala.collection.mutable.ArraysOps  and
scala.collection.mutable.IndexedSeq Transformation

scala.Array and all collections that implement the 
scala.collection.mutable.ArrayOps and scala.collection.mutable.IndexedSeq
traits, like scala.collection.mutable.ArraySeq,
scala.collection.mutable.StringBuilder, scala.collection.mutable.ArrayBuffer,
share the same transformation, since all of them are indexed *mutable*
sequences. The exact transformation is defined on lines XX-YY. As an example,
any macroMap applications on Arrays like:

Array(1, 2, 3).macroMap(_ + 1)

will be replaced from:

{
  def local1(x$1: Int): Int = x$1.+(1);
  val xs: Array[Int] = scala.this.Predef.intArrayOps(scala.Array.apply(1,
scala.this.Predef.wrapIntArray(Array[Int]{2, 3}))).repr();
  var buf: Array[Int] = new Array[Int](xs.length());
  var i: Int = 0;
  while(i.<(xs.length())){
    buf.update(i, local1(xs.apply(i))); // buf(i) = (local1(r(i)));
    i = i.+(1)
  };
  buf
}

Exactly the same transformation applies for all the other collections that fall
into this category. The reasons we chose this transformation path for this
category are:
- only this category's collections accept a length in the constructor
- the length method here takes constant time
- the element selection through xs.apply(i) takes constant time
- the update method is supported, since all collections of this category are
mutable

Also, all of this category's collections can only be mutated through the
update method (the assignment operator) on a specific index, with one
exception, the scala.collection.mutable.ArrayBuffer. ArrayBuffer supports
mutating methods for appending/removing elements which could affect the
semantics of macroMap in case we mutate the underlying collection in the
function we pass in macroMap. Our transformation seems to behave in the same
way as the default map method. For example both ``buf map {
x => buf += x; x+1 }'' and ``buf macroMap { x => buf += x;
x+1 }'' return ArrayBuffer(2, 3, 4). So, we assume the macroMap keeps the
same semantics on the ArrayBuffer as the original map method.

In the next chapter will see how much faster this version is compared to the
default map on Arrays.

Subsection: scala.collection.LinearSeq Transformation

This category includes scala.collection.immutable.List,
scala.collection.immutable.Stream, scala.collection.immutable.Queue,
scala.collection.immutable.Stack, scala.collection.mutable.MutableList,
scala.collection.mutable.MutableList, scala.collection.mutable.LinkedList,
scala.collection.mutable.DoubleLinkedList. All of these collections are linear.
For example, in this category's transformation we couldn't use the apply (for
indexing) or the length method because both of them take linear time, instead of
constant. The transformation is defined on lines XX-YY. As an example, any
macroMap applications on scala.collection.immutable.List like:

List(1, 2, 3).macroMap(_ + 1)

will be replaced from:

{
  private def local1(x$1: Int): Int = x$1.+(1);
  private def builder1: scala.collection.mutable.Builder[Int,List[Int]] = {
    val b: scala.collection.mutable.Builder[Int,List[Int]] =
immutable.this.List.canBuildFrom[Int].apply();
    b.sizeHint(immutable.this.List.apply[Int](1, 2, 3));
    b
  };
  val buf = builder1();
  var these = immutable.this.List.apply[Int](1, 2, 3);
  while(!these.isEmpty){
    buf.+=(local1(these.head));
    these = these.tail
  };
  buf.result
}

This category uses this transformation because:
- isEmpty takes constant time
- head takes constant time
- tail takes constant time

In the next chapter will see how much faster this version is compared to the
default map on Lists.

Subsection: scala.collection.Traversable Transformation

This category includes all colletions that don't fit in any of the previous
categories. Roughly it includes all kinds of Sets, Maps, Buffers and
immutable.IndexedSeqs. Obviously, since it includes such a broad range of
collections we are allowed to use only methods that are supported from all
collections, meaning we cannot use the length, apply, head, tail methods of the
previous transformations. The transformation is defined on lines XX-YY. As an
example, any macroMap applications on scala.collection.immutable.Set like:

Set(1, 2, 3).macroMap(_ + 1)

will be replaced from:

{
  private def local1(x$1: Int): Int = x$1.+(1);
  private def builder1:
scala.collection.mutable.Builder[Int,scala.collection.immutable.Set[Int]] = {
    val b:
scala.collection.mutable.Builder[Int,scala.collection.immutable.Set[Int]] =
immutable.this.Set.canBuildFrom[Int].apply();
    b.sizeHint(r);
    b
  };
  val buf = builder1();
  val it = r.toIterator;
  while(it.hasNext){
    buf.+=(local1(it.next));
  };
  buf.result
}


Here, we use the iterator method to generate a prefix's iterator object. All
collections have an iterator method since all of them implement the
scala.collection.Iterable trait. Each collection implements it using its
scala.collection.Traversable's foreach method, which is implemented according to
each collection's special characteristics in order to be more efficient.

For the collections that implement the immutable.IndexedSeq trait, like Vector,
we could have used a modified version of the mutable indexed sequences
transformation where we would use a builder object to build the target
collection, since we cannot mutate it in place. Interestingly, our experiments
showed us that solution was slower than the current one.