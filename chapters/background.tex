Section: The Problem

In 'Fixing The Inlining ``Problem''' [cite], Cliff Click describes an issue
that has emerged the last few years in the JVM ecosystem:

``The Problem is simply this: new languages on the JVM (e.g. JRuby) and new
programming paradigms (e.g. Fork Join) have exposed a weakness in the current
crop of inlining heuristics.  Inlining is not happening in a crucial point in
hot code exposed by these languages, and the lack of inlining is hurting
performance in a major way.  AND the inlining isn’t happening because The
Problem is a hard one to solve; (i.e. it’s not the case that we’ll wave our
wands and do a quick fix & rebuild HotSpot and the problem will go away).   John
Rose, Doug Lea and I all agree that it's a Major Problem facing JVMs with long
and far reaching implications.''

Let's see what The Problem is with a small example in a Java-like language. The
Problem is getting the right amount of context in hot inner loops – which also
contain a mega-morphic virtual call in the loop and not much else.  Here is a
naive implementation of a ``map'' method that applies a predetermined
function to all the elements of a source array and assigns the result to a
destination array. In this case we increment all source's elements by one:

// The function in the inner loop
long add1(long a) {return a + 1;}
// The iterator function
void map(long[] dst, long[] src) {
  for(int i=0; i < dst.len; i++) // simple loop
    dst[i] = add1(src[i]); // around a simple loop body
}

Inlining the function ``add1'' is crucial to performance here.  Without
inlining the compiler does not know what the loop body does (because function
calls can in general do anything), and with inlining it can understand the
entire function completely - and then see it’s a simple loop around a stream of
array references.  At this point the JIT can do range-check elimination, loop
unrolling, and prefetching, among other optimizations.

The Problem is that there are multiple variations on ``add1'' \emph{and}
the wrapping iterator gets complex.  It’s the product of these two parts getting
complicated that makes The Problem. In this work, we are mostly interested in
the first part, since the Scala collections iterators are not very big or
complex.

More often than not, after implementing map, we would like to add more
functions similar to add1. We might also want to add ``add2'', ``mult3'',
``filter'' and so on. What we really want is a way to pass in the function to
apply on the basic data bits in the innermost loop of our iterator.

In Java we often do this with either a Callable or a Runnable:
// A sample iterator function
void map( CallableOneArg fcn1arg, long[] dst, long[] src) {
  for(int i=0; i < dst.len; i++)
    dst[i] = fcn1arg.call(src[i]);
}
  
We need only 1 copy of our iterator, and we can apply nearly all kinds of one
argument functions. Alas, that inner loop now contains a function call that
needs inlining and there are dozens of different functions for ``fcn1arg.call''.
The JIT does not know which one to inline here, because all these different
functions are called at different times.  Typically then the JIT does not inline
any of them, and instead opts for a virtual call.  While the virtual call
itself is not too bad, the lack of knowledge of what goes on inside the virtual
call prevents all kinds of crucial optimizations: loop unrolling, range-check
elimination, all kinds of prefetching and alias analyses.

One solution  would be to make the inner function call a static (final)
call, then the JIT can inline. Of course, if we do that we need an iterator for
the ``add1'' version, one for the ``add2'' version, and one for the
``mutl3'' version, so we need a lot of them. Also, we will need a new one for
each new function we can think of; we cannot just name them all up front.  So we
will end up with a lot of these iterators each with a custom call in the inner
loop. All these iterators will start blowing out the instruction cache on our
CPU, and besides it is a pain to maintain dozens of cloned possibly complex
iterators.

Several Java ecosystem's prominent figures, like Cliff Click, John Rose, Doug
Lea, have proposed solutions that range from pure obscure technical to pure
educational ones, demonstrating possible ``megamorphic inlining friendly''
coding styles.

Scala, as most of the modern JVM languages, is no exception and it suffers from
The Problem too and, actually, it affects its adoption negatively. At the end of
2011, an email from a Yammer employee towards TypeSafe's CEO, about Scala
shortcomings, leaked to the public {cite
http://www.infoq.com/news/2011/11/yammer-scala}. Most complaints were related
with The Problem and, more specifically, with the Scala standard
library collections' performance. In the next chapter we will see where exactly
the problem lies and how Scala macros can help us alleviate it. Before that, it
would be helpful to overview the Scala language, its collections and its newly
added macros.


Section: Scala Overview

Scala \fxfatal{cite} is a relatively new statically typed programming language
that tries to unify the object-oriented and functional programming paradigms
into one coherent paradigm, recently called object-functional. Currently its
main implementation runs on the JVM and so its main goal is to provide a more
general and uniform superset of Java. Since version 2.8, Scala has a rich
collections library and since version 2.10 it has a completely new reflection
subsystem.

Subsection: Scala Collections Overview

The Scala library systematically distinguishes between mutable and immutable
collections. A mutable collection can be updated or extended in place. This
means one can change, add, or remove elements of a collection as a side effect.
Immutable collections, by contrast, never change. We still have operations
that simulate additions, removals, or updates, but these operations will in each
case return a new collection and leave the old collection unchanged. We will see
in the next chapter how this mutable-immutable separation affects our macro
transformation plan.

All collection classes are found in the package scala.collection or one of its
sub-packages mutable, immutable, and generic. Most collection classes needed by
client code exist in three variants, which are located in packages
scala.collection, scala.collection.immutable, and scala.collection.mutable,
respectively. Each variant has different characteristics with respect to
mutability.

A collection in package scala.collection.immutable is guaranteed to be immutable
for everyone. Such a collection will never change after it is created.
Therefore, one can rely on the fact that accessing the same collection value
repeatedly at different points in time will always yield a collection with the
same elements.

A collection in package scala.collection.mutable is known to have some
operations that change the collection in place. So dealing with mutable
collections means one needs to understand which code changes which collection
when.

A collection in package scala.collection can be either mutable or immutable. For
instance, collection.IndexedSeq[T] is a superclass of both
collection.immutable.IndexedSeq[T] and collection.mutable.IndexedSeq[T]
Generally, the root collections in package scala.collection define the same
interface as the immutable collections, and the mutable collections in package
scala.collection.mutable typically add some side-effecting modification
operations to this immutable interface.

The difference between root collections and immutable collections is that
clients of an immutable collection have a guarantee that nobody can mutate the
collection, whereas clients of a root collection only promise not to change the
collection themselves. Even though the static type of such a collection provides
no operations for modifying the collection, it might still be possible that the
run-time type is a mutable collection which can be changed by other clients.

By default, Scala always picks immutable collections. For instance, if we just
write Set without any prefix or without having imported Set from somewhere, we
get an immutable set, and if we write Iterable we get an immutable iterable
collection, because these are the default bindings imported from the scala
package. To get the mutable default versions, we need to write explicitly
collection.mutable.Set, or collection.mutable.Iterable.

The following figure shows all collections in package scala.collection. These
are all high-level abstract classes or traits, which generally have mutable as
well as immutable implementations. (figure from ...)

[figure]

The following figure shows all collections in package
scala.collection.immutable. (figure from ...)

[figure]

And the following figure shows all collections in package
scala.collection.mutable. (figure from ...)

[figure]

This work focuses mainly on Seq's subtree, since it's where we can get the
most prominent speedups by exploiting the sequences' properties. A sequence is a
kind of iterable that has a length method and whose elements have fixed index
positions, starting from 0.


Subsection: Scala Compile-Time Reflection Overview

Scala version 2.10, released on \fxfatal{date}, introduced a new reflection
subsystem adding both run time and compile metaprogramming capabilities. The new
run-time reflection is much more general and feature complete compared to the
Java's reflection. Compile-time reflection is quite rare in the mainstream
statically typed programming languages and, currently, it can only be found in
more exotic functional languages like Haskell \fxfatal{cite Template Haskell}
and ML \fxfatal{cite Meta-ML}. Compile-time reflection enabled the introduction
of an experimental version of type-safe syntactic macros {cite Scala Macros, a
Technical Report}. 

Syntactic macro systems work at the level of abstract syntax trees and preserve
the lexical structure of the original program. Macro systems that work at the
level of lexical tokens, like the C preprocessor, cannot preserve the
lexical structure reliably. The most widely used implementations of syntactic
macro systems are found in Lisp-like languages such as Common Lisp, Scheme,
ISLISP and Racket [cite each language]. These languages are especially suited
for this style of macro due to their uniform, parenthesized syntax (known as
S-expressions).

Compile-time metaprogramming is a valuable tool for enabling such programming
techniques as:

- Language virtualization (overloading/overriding semantics of the original
programming language to enable deep embedding of DSLs),
- Program reification (providing programs with means to inspect their own code),
- Self-optimization (self-application of domain-specific optimizations based on
program reification),
- Algorithmic program construction (generation of code that is tedious to write
with the abstractions supported by a programming language).

This work falls in the third category, since we use the compile-time reflection
to optimize the Scala collections.

Scala's macro system allows programmers to write macro defs: functions that
are transparently loaded by the compiler and executed during compilation. This
realizes the notion of compile-time metaprogramming for Scala users.

Our project is implemented directly in the Scala compiler (scalac), so we can
use most of the available compile-time metaprogramming capabilities directly,
without using macro defs explicitly. In the next chapter we will see how we
achieve it.

The compile-time reflection allows us to create new and/or manipulate
existing abstract-syntax trees (ASTs) during compiler's typechecking phase.
All the new or changed ASTs are re-typechecked guaranteeing us type-safe
transformations. scalac represents ASTs with objects of type
scala.reflect.api.Tree or scala.reflect.api.Exprs, which is just a typed-wrapper
of scala.reflect.api.Tree. Through the available reflection APIs we can create,
inspect or change compiler's scala.reflect.api.Symbols and
scala.reflect.api.Types objects that are related with these ASTs.

For example, we can create the AST of the Scala expression ``x < 10'' manually,
either with a macro or directly within the scalac, with this code
``Apply(Select(Ident(newTermName("x")), newTermName("\$less"),
List(Literal(Constant(10)))))''. Apply, Select, Ident, Literal, Constant are AST
objects themselves of scala.reflect.api.Tree type.

Obviously the AST construction is cumbersome and error-prone. But most probably
it is also wrong. If the AST was generated within an internal scalac method or
within a macro, the returned AST will be inlined and type-checked at the
method/macro call site. But this means that the identifier ``x'' will be
type-checked at a point where it is most likely not visible, or in the worst
case they might refer to something else. In the macro literature, this
insensitivity to bindings is called non-hygienic {cite [19, 8]}. Scala's
compile-time reflection solves the non-hygiene problem providing a built-in
macro, called reify, that produces its tree one stage later.

The reify macro plays a crucial role in the compile-time metaprogramming. Its
definition as a member of Context is:

def reify[T](expr : T): Expr[T] = macro . . .

Reify accepts a single parameter expr, which can be any well-typed Scala
expression, and creates a tree that, when compiled and evaluated, will recreate
the original tree expr. So reify is like time-travel: trees get re-constituted
at a later stage. If reify is called from normal compiled code, its effect is
that the AST passed to it will be recreated at run time.
Consequently, if reify is called from a macro implementation or a method inside
scalac, its effect is that the AST passed to it will be recreated at
macro-expansion time (which corresponds to run time for macros). This gives a
convenient way to create syntax trees from Scala code: pass the Scala code to
reify, and the result will be a syntax tree that represents that very same code.

For example, ``reify(x < 10)'' will generate an Expr object representing the
same AST we created manually before.

More importantly, reify packages the result expression tree with the types and
values of all free references that occur in it. This means in effect that all
free references in the result are already resolved, so that re-typechecking the
tree is insensitive to its environment. All identifiers referred to from
an expression passed to reify are bound at the definition site, and not re-bound
at the call site. As a consequence, macros that generate trees only by means
of passing expressions to reify are hygienic.

So, in a sense, Scala macros are self-cleaning. Their basic form is minimal
and unhygienic, but that simple form is expressive enough to formulate a
reify macro, which in turn can be used to make tree construction in macros
concise and hygienic.

Another important compile-time metaprogramming operation is the ``splicing'',
which could be described as reify's inverse operation. Using Expr's splice
method we can inject an existing AST inside a reify's body.

Reification and splicing operations are crucial to our implementation, which we
will see in the next chapter.