\label{related}
As discussed earlier, Scala's performance issue-s-, due to the semantic gap between high-level abstractions and the runtime model of the JVM, are not new, so a lot of research is taking place in order to improve its performance. Dragos' PhD thesis~\cite{dragos2010compiling} was dedicated to the Scala compiler's optimizations and specialization. More related to our work was~\cite{dragos2008optimizing}, where he applied a series of more aggressive optimizations of higher-order functions in the scalac, through decompilation of library code combined with inlining, dead code elimination, and copy propagation.

More recently, and in parallel with the new Scala compile-time code generation facilities we saw in this work, a new dynamic code generation approach has emerged, called \emph{Lightweight Modular Staging} (LMS)~\cite{rompf2010lightweight,rompf2011building}. This framework provides a library of core components for building high performance code generators and embedded compilers in Scala, enabling the creation of new DSLs/optimizations that improve the performance of Scala and its libraries~\cite{moors2012scala,brown2011heterogeneous,ureche2012stagedsac,rompf2013optimizing}.

Another interesting technique for collections performance improvement, which has been applied to Haskell's reference compiler, the Glasgow Haskell Compiler (GHC), is \emph{Stream Fusion}~\cite{coutts2007stream,mainlandhaskell}. It would be interesting to see how it applies in Scala's case too.
